\documentclass{exam}

\newcommand\myroot{../..}
\usepackage[hw]{\myroot/course}

\title{Final Project: Coronavirus version}
\author{\usnaInstructorShort}
\date{\today}
\duedate{TBD}

\begin{document}
\maketitle
\begin{abstract}
With the COVID-19 situation we will have to scale back our planned \usnaCourseNumber\ projects. The original intent was for you to apply the techniques you have learned over the semester to answer a research question of your choosing, developing testable hypotheses, devising methods, and using the results to evaluate the hypotheses. Think flexible and adaptable; the key metrics for the remaining time are that you have fun and learn something. 
\end{abstract}

\begin{questions}
\question Find something around your home or in your community that you find biomechanically interesting. Research it online, and observe it directly.  

\question Develop a research question that you want to answer, preferably one that has several testable hypotheses, alternatives that you could decide among based on a well designed experiment. 

\question Use your creativity and improvise methods based on what you learned in class. Biomechanics and field ecologists are notorious for improvising methods with minimal supplies and funding; for example, adapting video cameras or tape recorders to collect data; using buckets, water, and rocks to provide known forces or pressures; using mirrors or the sun to mitigate having only one camera; developing experimental rigs from trash, stuff from the garage, or stuff from the hardware store. People have made models out of sculpey, flow meters out of chalk, instron machines out of a bucket and a know, algae models out of flagging tape... 

\question Try out several methods before committing to something, and use the tries to improve, mix and match, or prototype/develop other ideas. Then, when you are ready, try to get a test data set to try out your analyses. 

\question Methods never survive first contact with your experimental organism. Time permitting, use what you learned to go again and try to collect as complete a dataset as time and circumstances allow. The hope is it will be enough for your project, or to serve as fodder for a feasibility or pilot study leading to a proposal and funding; or if you're really lucky you'll have enough for a publication. 

\question Several example peer reviewed biology journal articles will be provided for you. Try to frame what you did in a similar fashion including abstract; introduction; methods and materials; results; and discussion. When you make plots do your best to make them look right, and keep all files needed to re-make them; you don't need to waste time sticking them in your document just keep them at the end or as separate files for now. 
\end{questions}
\end{document}
